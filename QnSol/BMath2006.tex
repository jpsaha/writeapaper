\begin{example}[\href{https://artofproblemsolving.com/community/c6h484358p2713710}{India BMath 2006}]
\index{India BMath!2006}
\label{India BMath 2006}
A domino is a $2$ by $1$ rectangle. For what integers $m$ and $n$, can one cover an $m$ by $n$ rectangle with non-overlapping dominoes?
\end{example}

\walkthrough
\begin{walk}
		\ii If an $m\times n$ rectangle admits a covering by non-overlapping dominos, then show that at least one of the integers $m,n$ has to be even. 
		\ii If at least one of $m,n$ is even, then prove that an $m\times n$ rectangle admits a covering by non-overlapping dominos.
\end{walk}
	
\begin{figure}
\centering
	\begin{asy}
		size(5cm);
		real c = 0.03;
		pair P = (c, c); 
		for (int i=0; i < 8; ++i)
		 {for (int j=0; j< 5; ++j)
		 draw((i,j)--(i+1,j)--(i+1,j+1)--(i, j+1)--cycle, royalblue+1);
		 }
path hbox = box((0,0)+P,(2,1) -P);
path vbox = box((0,0)+P,(1,2) -P);
fill(hbox, mediumgray);
fill(shift(2,2)*hbox, mediumgray);
fill(shift(1,3)*hbox, mediumgray);
fill(shift(1,4)*hbox, mediumgray);
fill(shift(3,0)*hbox, mediumgray);
fill(shift(3,1)*hbox, mediumgray);
fill(shift(4,4)*hbox, mediumgray);
fill(shift(5,3)*hbox, mediumgray);
fill(shift(5,0)*hbox, mediumgray);
fill(shift(6,4)*hbox, mediumgray);
fill(shift(2,0)*vbox, lightgray);
fill(shift(0,1)*vbox, lightgray);
fill(shift(1,1)*vbox, lightgray);
fill(shift(0,3)*vbox, lightgray);
fill(shift(3,3)*vbox, lightgray);
fill(shift(4,2)*vbox, lightgray);
fill(shift(5,1)*vbox, lightgray);
fill(shift(5,1)*vbox, lightgray);
fill(shift(7,0)*vbox, lightgray);
fill(shift(7,2)*vbox, lightgray);
fill(shift(6,1)*vbox, lightgray);
//draw(shift(.5,.5)*thebox,green+linewidth(5pt));
//clip(thebox);
//draw(shift(-.5,-.5)*thebox,red+linewidth(5pt));
	\end{asy}
\caption{India BMath 2006 (a tiling of a $5\times 8$ rectangle with non-overlapping dominoes)}
\label{Com Fig: India BMath 2006}
\end{figure}

\begin{soln}
In the following, an $m\times n$ rectangle is to be thought as an $m\times n$ rectangular grid. 

To be able to cover an $m\times n$ rectangle by non-overlapping dominoes, it is necessary for the product $mn$ to be even, and hence, at least one of $m, n$ is even. 
Indeed, if an $m\times n$ rectangle admits a covering using $k$ non-overlapping dominoes, then those dominoes together cover $2k$ unit squares, 
and this yields that $2k = mn$. 

Moreover, when at least one of $m, n$ is even, an $m\times n$ rectangle can be covered by non-overlapping dominoes by 
covering each row by $m/2$ (resp. each column by $n/2$) non-overlapping dominos if $m$ (resp. $n$) is even. 

This shows that an $m\times n$ rectangle can be covered by non-overlapping dominoes if and only if at least one of $m,n$ is even. 
\end{soln}

\begin{remark}
The above conclusion shows that 
an $m\times n$ rectangle admits a covering by non-overlapping dominoes if and only if 
it admits a covering by non-overlapping dominoes in the \emph{most obvious manner}, i.e. a covering by non-overlapping dominoes such that all of them are either horizontal or vertical
(cf. \cite[p. 6]{BrualdiIntroCombi}). 
\end{remark}

The following problem is a more general version of \cref{India BMath 2006}. 

\begin{example}
\cite[Problem 8, Chapter 2, p. 26]{EngelProblemSolvingStrategies}
Show that an $m\times n$ rectangle admits a covering by non-overlapping $k\times 1$ rectangles if and only if $k$ divides $m$ or $k$ divides $n$. 
\end{example}
